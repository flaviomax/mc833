\documentclass[12pt,a4paper]{article}

\usepackage[brazil]{babel}
\usepackage[utf8]{inputenc}
\usepackage[T1]{fontenc}
\usepackage{graphicx}
\usepackage{epsfig}
\usepackage{cite}
\usepackage{verbatim}
\usepackage{concrete}
\usepackage{amsfonts}
\usepackage{amsmath}
\usepackage{amssymb}
\usepackage{setspace}
\usepackage{url}
\usepackage{subfigure}
\usepackage{array}
\usepackage{multirow}
\usepackage{colortbl}
\usepackage{tabularx}
\usepackage{geometry}
\usepackage{cite}
\usepackage{xcolor}
\usepackage{color}
\usepackage{url}
\usepackage{lmodern}
\usepackage{pict2e}
\usepackage{siunitx}

\geometry{top=2.3cm,bottom=2.2cm,left=2.5cm,right=2.5cm}

\newcommand{\V}[1]{\boldsymbol{#1}}

\providecommand{\todo}[1]{{\textbf{\protect\color{red} {TODO: #1}}}}

\newenvironment{myitemize}{
\begin{itemize}
 \setlength{\itemsep}{1pt}
 \setlength{\parskip}{0pt}
 \setlength{\parsep}{0pt}
}{\end{itemize}}

\newenvironment{myenumerate}{
\begin{enumerate}
 \setlength{\itemsep}{1pt}
 \setlength{\parskip}{0pt}
 \setlength{\parsep}{0pt}
}{\end{enumerate}}

\include{abaco}

\begin{document}

\tolerance = 999
\sloppy

\hyphenation{re-co-nhe-ci-men-to}
\hyphenation{ca-rac-te-rís-ti-cas}
\hyphenation{a-pre-sen-ta-ção}
\hyphenation{pos-si-bi-li-ta-rá}

\thispagestyle{empty}

\begin{center}{\large \bf UNIVERSIDADE ESTADUAL DE CAMPINAS} \end{center}
\begin{center}{\large INSTITUTO DE COMPUTAÇÃO} \end{center}

\begin{center}
\begin{minipage}[tl]{31mm}
  \ABACO{1}{9}{6}{9}{1}
\end{minipage}
\end{center}

\vspace*{3.5cm}

\begin{center}
{\Large \bf MC833 - Relatório Científico 3}

\vspace*{2.0cm}

\textsc{\Large Comunicação Cliente-Servidor} \\ [0.1cm]
\textsc{\Large Usando Protocolo RMI}

\vspace{2.5cm}

\textbf{135368:} Cristina Freitas Bazzano \\
\textbf{135749:} Flávio Altinier Maximiano da Silva

\vspace{7.5cm}

{\large Campinas - SP}
\vspace*{0.2cm}

{\large Junho de 2015}
\end{center}

\clearpage

\pagenumbering{arabic}
\onehalfspacing
\tableofcontents 
\clearpage
\onehalfspacing

\begin{abstract}
Este trabalho focou-se no estudo temporal de redes cliente-servidor baseadas em comunicação utilizando Java RMI. Foi implementado um serviço de locadora de filmes baseado em {\it MySQL} no servidor, e ao cliente foram adicionadas diversas operações de acesso ao banco, baseando-se em operações de consultas pequenas, grandes e de escrita. Observou-se que a escrita no banco é o processo mais lento, enquanto operações de consultas extensas são apenas brevemente mais lentas que buscas mais curtas. Comparamos os resultados de tempo desse processo utilizando RMI com os obtidos em um trabalho anterior baseado em TCP diretamente programado na linguagem C. Apesar de ser esperado que as operações em RMI fossem mais lentas, observamos exatamente o contrário; culpamos esse fato à melhor adaptação do MySQL ao Java do que ao C. Além disso, realizamos uma breve análise do tamanho de código e do nível de abstração.
\end{abstract}

\section{Introdução}
Na Internet, a grande maioria das montagens de comunicação baseia-se em um sistema cliente-servidor. Nesse contexto, o servidor guarda alguma informação que é relevante ao cliente, que pede essa informação através de uma mensagem na rede; o servidor, por sua vez, responde esse pedido com outra(s) mensagem(ns) contendo a informação requisitada.

O método de comunicação escolhido para este projeto é o {\it Remote Method Invocation} (RMI)~\cite{downing1998java, RMITutorial}. Esse método, desenvolvido utilizando a linguagem Java, é baseado em comunicação TCP~\cite{postel1981transmission} (sem perda de pacotes) e baseia-se em um host (o cliente) ser capaz de invocar métodos do outro (servidor) como se fossem métodos locais. Toda a comunicação TCP é feita ''escondida'' pelo sistema.

Neste trabalho, num primeiro momento, o foco principal de nossa análise será a medida de tempo total de comunicação entre cliente e servidor, além de uma aproximação do tempo de transmissão das mensagens entre os dois (tempo de as mensagens saírem de um host e chegarem ao outro). Faremos também uma comparação desses tempos com os obtidos em um trabalho anterior, no qual estudamos a implementação direta do protocolo TCP em C. O tamanho de código será também analisado, além de um breve parecer sobre o nível de abstração do sistema desenvolvido.

\section{Descrição Geral do Sistema}

O sistema construído consiste em uma simples montagem de cliente-servidor, com comunicação utilizando RMI. 

    O sistema montado é uma emulação de um serviço de locadora online, onde o servidor contém diversas informação sobre os filmes que possui em seu estoque, os quais podem ser acessadas por clientes ou pelo cliente administrador da locadora. 
    
    Para sintetizar o problema da melhor forma possível, criamos no nosso banco de dados uma tabela “Locadora” que armazena os seguintes atributos sobre os filmes que possui: título, ano de lançamento, gênero, duração, sinopse, diretor, ator principal, ator coadjuvante principal, exemplares e um id único para cada filme. 
    
Os acessos do cliente ou do administrador da locadora ao servidor são com pedidos de informação ou escrita de dados de cada filme, sendo eles da seguinte forma:

\begin {myenumerate}
\item Listar todos os títulos dos filmes e o ano de lançamento;
\item Listar todos os títulos dos filmes e o ano de lançamento de um gênero determinado;
\item Dado o identificador de um filme, retornar a sinopse do filme;
\item Dado o identificador de um filme, retornar todas as informações deste filme;
\item Listar todas as informações de todos os filmes;
\item Alterar o número de exemplares em estoque;
\item Dado o identificador de um filme, retornar o número de exemplares
em estoque.
\end{myenumerate}

    Todos esses acessos podem ser feitos em dois modos de consulta: um que realiza a consulta apenas uma vez, para testar sua corretude, e outro que realiza a consulta sequencialmente, cada acesso é repetido 30 vezes e todos os tempos são armazenados em um arquivo {\it log}, para possibilitar uma futura análise dos mesmos.
    
    Classificamos esses acessos em 3 categorias: as consultas pequenas, consultas grandes e operações de escrita. De acordo com a consulta que cada operação realiza podemos dividi-las da seguinte forma: no primeiro grupo se encontram as operações de 1 a 4 e a operação 7, no segundo grupo temos apenas a operação 5 e no terceiro grupo temos a operação 6. 
    
O primeiro grupo é caracterizado por uma consulta rápida de até 1500 bytes, que é o tamanho máximo MTU (Maximum Transmission Unit) que um pacote pode ter para ser transmitido pela camada Ethernet. O segundo grupo realiza consultas longas, com mais de 1500 bytes, que devem ser quebradas em mais de um pacote para serem enviadas. E o terceiro grupo é caracterizado por uma consulta que carrega um maior atraso do banco de dados, uma vez que no tempo medido está embutido o tempo de escrita no banco.


\section{Armazenamento e Estruturas de Dados}

Em um sistema real de cliente-servidor os dados são em sua maioria armazenados em Banco de Dados para garantir a persistência e a consistência dos mesmos. Com o objetivo de nos aproximar o máximo possível de um problema real, adotamos o MySQL~\cite{mysql1995mysql} como o Banco de Dados para armazenar os dados da locadora de filmes no servidor.

    Populamos o banco do servidor com dados de 10 filmes retirados do site IMDb~\cite{IMDbsite}, os quais estão classificados com as 10 melhores notas no mesmo. Optamos por uma quantia pequena de exemplares para que os tempos medidos nas operações não fossem totalmente influenciados pelo tempo de acesso ao servidor, visto que os tempos de comunicação são o foco da análise do projeto.
    
    No acesso ao banco de dados, os clientes podem apenas acessar a informação, enquanto o cliente administrador da locadora pode, além do acesso, alterar a quantidade de exemplares em estoque. Para garantir esse privilégio de acesso, é necessário uma senha para entrar no modo administrador, a qual é processada pelo programa e só após confirmada o acesso diferenciado ao banco pode ocorrer.
    
Como o RMI utiliza por baixo o protocolo TCP na comunicação cliente-servidor, não fizemos nenhuma manobra para comprovar sua confiabilidade, umas vez que já comprovamos a confiabilidade do protocolo TCP em um trabalho passado, e portanto, podemos afirmá-la para o RMI.

\section{Detalhes de Implementação}

Tanto o cliente quanto a servidor foram implementados baseando-se nas descrições encontradas em~\cite{RMITutorial}. Porém note que métodos do cliente não são necessários estar registrados em RMI, já que ele apenas precisa consultar o servidor (e não o contrário).

Por esse motivo, optamos por manter apenas as operações do servidor em sua pasta pública. Além disso, como no nosso sistema apenas strings são enviadas e recebidas, pudemos remover totalmente a classe {\it Task.java} e tornar a interface {\it Compute.java} menos robusta: agora, ele recebe uma string como parâmetro (a query SQL) e retorna outra string (o resultado da operação).

O servidor é iniciado em uma máquina utilizando todas as operações demonstradas no Tutorial~\cite{RMITutorial}. Como especificado, sua interface RMI é apenas a função Compute (que recebe uma query SQL), cuja função é executar a query no banco de dados de filmes e retornar os resultados.

O cliente, a cada vez que é iniciado, joga na tela um {\it prompt} com as opções de requisição ao servidor. Em seguida, após escolhida a opção, o cliente dá a opção de fazer o acesso apenas uma vez (para verificar a corretude da aplicação) ou trinta vezes. Se o usuário escolher por fazer o acesso apenas uma vez, o resultado da consulta é exibido na tela do cliente. Se escolher por fazer a consulta trinta vezes, um arquivo de {\it log} de respostas é criado na mesma pasta do cliente, com o nome {\it clientTime.txt}.

A consulta é feita ao servidor pelo meio de envio de consultas {\it MySQL}, que são montadas no próprio programa do cliente. Essas consultas são enviadas como parâmetros da função Compute, presente na interface RMI do servidor, que simplesmente as aplica ao seu banco de dados e retorna o resultado das consultas de leitura ou escrita, como retorno da função Compute.

Esse arquivo de {\it log} gerado tem o seguinte formato: são trinta linhas e uma coluna, onde cada linha representa um dos testes. Os valores numéricos são os tempos, em nanossegundos, de cada chamada à função Compute pelo cliente.

O servidor também concatena a um arquivo de {\it log} dados sobre cada conexão que coordena. Como o servidor concatena informações num mesmo arquivo, quando o cliente fizer, por exemplo, trinta consultas de títulos de filmes, as últimas trinta entradas desse {\it log} do servidor guardarão os dados sobre essas consultas. Cada entrada nesse {\it log} é um {\it inteiro} que guarda o tempo de acesso do servidor ao banco de dados; esse tempo é o intervalo desde o início da execução da função Compute até imediatamente antes de seu retorno. Dessa forma, podemos aproximar o tempo de transmissão das mensagens: usando o tempo $ tServer $ desse {\it log} do servidor e a segunda coluna do {\it log} do cliente ($ tClient $), o tempo de transmissão de uma mensagem ao outro será, aproximadamente, dado por:
\begin{equation}
{\text {Tempo de Transmissão}} = \frac{tClient - tServer}{2}
\label{eqn:time}
\end{equation}
Um outro pequeno detalhe de implementação é que apenas o cliente administrador da locadora deve ter autorização para alterar o número de exemplares em estoque de cada filme. Portanto, quando um cliente decide fazer essa alteração, é perguntado por uma senha. Essa senha é verificada no próprio programa do cliente (é hard-coded como “1234”) e, apenas se estiver correta, o programa continua. Essa escrita no banco de dados também é hard-coded para agilizar os testes (não é necessário perguntar o novo estoque a cada uma das trinta execuções.)

\section{Análise dos Resultados}

Toda a comunicação foi feita em LAN, com cliente e servidor conectados a um roteador D-Link DIR-600 por Wi-fi.

\subsection{Tempo total de Comunicação}

Em todos os resultados dos tempos médios utilizamos um intervalo de confiança com nível de confiança de 95\%, obtido a partir da soma ou subtração de dois desvios padrão.

Temos nesta seção os resultados dos experimentos usando o protocolo RMI; temos também os resultados para os mesmos experimentos usando TCP, de um relatório feito anteriormente, e esses dados serão comparados.
    
Primeiramente geramos um gráfico com o tempo total de conexão com o servidor para cada uma das operações. O resultado segue demonstrado na Tabela~\ref{table:totalR} e na Figura~\ref{fig:totalR}.

\begin{table}[h]
\centering
\caption{Tabela mostrando o tempo médio e intervalo de confiança para cada uma das sete operações, na ordem em que foram descritas, para o protocolo RMI. Tempo calculado no cliente, entre a execução da {\it task}.}
\label{table:totalR}
\begin{tabular}{lll}
Operação & Tempo Médio Total (\SI{}{\milli\second})  & Intervalo de Confiança (\SI{}{\milli\second})           \\
1        & 14.66019          & 5.715058    \\
2        & 19.90590          & 47.649582   \\
3        & 12.63895          & 7.884285    \\
4        & 13.66040          & 4.993054    \\
5        & 18.04669          & 7.716191     \\
6        & 72.64021          & 127.971248  \\
7        & 13.09539          & 7.597693   
\end{tabular}
\end{table}

Observe também na Tabela~\ref{table:totalT} e na Figura~\ref{fig:totalT} os resultados de tempo total de conexão, desde o {\it connect}() até o {\it close}(), para o protocolo TCP, para os mesmos experimentos.

\begin{figure}[h]
\centering
\includegraphics[width=\textwidth]{resultadosTempoTotalRMI.png}
\caption{Mostra o tempo médio e o intervalo de confiança de 95\% para as sete operações, na mesma ordem em que foram descritas, utilizando-se o protocolo RMI. Tempo calculado no cliente, entre a execução da {\it task}.}
\label{fig:totalR}
\end{figure}

\begin{table}[h]
\centering
\caption{Tabela mostrando o tempo médio e intervalo de confiança para cada uma das sete operações, na ordem em que foram descritas, para o protocolo TCP. Tempo calculado no cliente, desde a abertura até o fechamento da conexão.}
\label{table:totalT}
\begin{tabular}{lll}
Operação & Tempo Médio Total (\SI{}{\milli\second}) & Intervalo de Confiança (\SI{}{\milli\second})\\ \hline
1        & 38.42           & 5.59   \\
2        & 37.38           & 9.14   \\
3        & 37.47           & 7.59   \\
4        & 37.10           & 10.37  \\
5        & 40.41           & 9.15   \\
6        & 114.10          & 95.27  \\
7        & 36.49           & 7.88  
\end{tabular}
\end{table}

\begin{figure}[h]
\centering
\includegraphics[width=\textwidth]{resultadosTempoTotalTCP.png}
\caption{Mostra o tempo médio e o intervalo de confiança de 95\% para as sete operações, na mesma ordem em que foram descritas, para o protocolo TCP. Tempo calculado no cliente, desde a abertura até o fechamento da conexão.}
\label{fig:totalT}
\end{figure}

Os resultados que obtivemos para o tempo total de conexão dependem bastante das consultas ao banco de dados e do tamanho das mensagens. Essas mensagens nas operações pequenas, como já especificado, podem ser totalmente enviadas em apenas um pacote, enquanto nas operações grandes, vários pacotes são necessários para o envio completo das mensagens. Nas operações de escrita, apesar de só um pacote ser necessário para o envio da mensagem, a consulta ao banco de dados para alteração requer um tempo bem maior de operação.

Com isso em mente, os resultados foram neste sentido como esperado. Observe que tanto no TCP quanto no RMI as operações de escrita notoriamente tomam mais tempo que as outras. É notável também que as operações grandes levam um pouco mais de tempo que as pequenas - mas ainda têm tempos bem inferiores às de escrita.

O fato a se notar, entretanto, é a escala vertical dos gráficos: embora eles sejam parecidos, em todas as operações o TCP tomou mais tempo que o RMI. Em todos os experimentos realizados, essa afirmação mostrou-se verdadeira, com o tempo de comunicação via RMI sendo cerca de 2 vezes menor que via TCP. 

Entretando, era esperado que o RMI fosse mais lento que o TCP puro, uma vez que ele roda em cima do TCP (e de HTTP), escondendo sua complexidade. Acreditamos que este fato tenha ocorrido por causa das diferenças nas linguagens utilizadas nos dois projetos. Enquanto o RMI é implementado em Java, que possui um ótimo suporte para JDBC, o TCP foi implementado na linguagem C, sem o mínimo de suporte à Banco de Dados, e portanto, acreditamos que o acesso ao banco foi um gargalo bem maior no projeto com o TCP.

\subsection{Tempo de Transmissão}

Em seguida geramos um gráfico com o tempo total de transmissão das mensagens para cada uma das operações, como explicitado pela equação~\ref{eqn:time}. O resultado para o RMI segue na Tabela~\ref{table:transmissaoR} e Figura~\ref{fig:transmissaoR}. Na Tabela~\ref{table:transmissao} e Figura~\ref{fig:transmissao} temos os resultados para os mesmos experimentos utilizando o protocolo TCP.

\begin{table}[h]
\centering
\caption{Mostra o tempo médio e intervalo de confiança de 95\% para transmissão de mensagens segundo a equação~\ref{eqn:time} para cada uma das operações, na ordem em que foram descritas, utilizando o protocolo RMI.}
\label{table:transmissaoR}
\begin{tabular}{lll}
Operação & Tempo Médio de Transmissão (\SI{}{\milli\second}) & Intervalo de Confiança (\SI{}{\milli\second}) \\ \hline
1 & 1.544082 & 0.7207669  \\
2 & 4.410626 & 23.1782371 \\
3 & 1.465263 & 0.8149761  \\
4 & 1.666445 & 1.0031879  \\
5 & 2.141733 & 1.8006986  \\
6 & 7.834579 & 63.1107935 \\
7 & 1.509789 & 1.3315485 
\end{tabular}
\end{table}

Os tempos de transmissão medem o tempo de comunicação entre o cliente e o servidor. Esse tempo deve depender apenas do tamanho das mensagens, que como já especificado, corresponde apenas a um pacote nas operações pequenas e de escrita e de vários pacotes nas operações grandes. 

Os tempo obtidos para as operações pequenas e grandes condizem com essa expectativa, tanto no TCP quanto no RMI; no TCP as operações de escrita em alguns experimentos tiveram tempos de transmissão muito maiores (observe o alto desvio): acreditamos que isso se deva às aproximações no cálculo do tempo de transmissão. Para o RMI também podemos observar que as operações de escrita estão acima das demais, um resultado condizente, uma vez que ele roda em cima do TCP; operações grandes precisaram de um intervalo de tempo um pouco maior para serem transmitidas.

Note que, diferente do esperado, os tempos de transmissão das mensagens RMI foram muito inferiores aos tempos do TCP;  enquanto uma operação grande demorou cerca de $\SI{10}{\milli\second}$, em média, para ser transmitida no TCP, no RMI levou apenas $\SI{2}{\milli\second}$, em média. Como comentado na seção anterior, atribuímos esse resultado as diferentes linguagens utilizadas.

\begin{figure}[h]
\centering
\includegraphics[width=\textwidth]{resultadosTempoTransmissaoRMI.png}
\caption{Tempo médio, com intervalo de confiança de 95\%, calculado conforme a equação~\ref{eqn:time}, para as sete operações na mesma ordem em que foram descritas, utilizando o protocolo RMI.}
\label{fig:transmissaoR}
\end{figure}

\begin{table}[h]
\centering
\caption{Mostra o tempo médio e intervalo de confiança de 95\% para transmissão de mensagens segundo a equação~\ref{eqn:time} para cada uma das operações, na ordem em que foram descritas, utilizando o protocolo TCP.}
\label{table:transmissao}
\begin{tabular}{lll}
Operação & Tempo Médio de Transmissão (\SI{}{\milli\second}) & Intervalo de Confiança (\SI{}{\milli\second}) \\ \hline
1        & 11.02                & 2.86   \\
2        & 10.35                & 4.09   \\
3        & 10.79                & 3.30   \\
4        & 10.43                & 4.61   \\
5        & 11.99                & 4.45   \\
6        & 24.19                & 48.88  \\
7        & 10.37                & 3.66  
\end{tabular}
\end{table}

\begin{figure}[h]
\centering
\includegraphics[width=\textwidth]{resultadosTempoTransmissaoTCP.png}
\caption{Tempo médio, com intervalo de confiança de 95\%, calculado conforme a equação~\ref{eqn:time}, para as sete operações na mesma ordem em que foram descritas, utilizando o protocolo TCP.}
\label{fig:transmissao}
\end{figure}

\subsection{Tamanho de Código}

As operações de redes são, na verdade, parte pequena do código total da aplicação. No servidor, a parte mais extensa do código são claramente as operações de consulta {\it MySQL}. No cliente, além das operações de banco de dados, há dezenas de linhas voltadas apenas para a saída na tela dos {\it prompts} ou a escrita em arquivo.

As operações de redes, mesmo no servidor que faz algumas operações a mais que o cliente, representam muito pouco do código. Considerando-se o servidor, as operações de redes mais os seus testes de erros somam 51 linhas, enquanto o arquivo todo tem 191 (correspondem a cerca de 26\%). Para o cliente, as operações de redes representam 31 de um total de 342 linhas (cerca de 9\%). 

\subsection{Confiabilidade}

Como o protocolo escolhido foi o RMI o qual roda em cima do protocolo TCP, que garante confiabilidade na transmissão de mensagens, fato já comprovado experimentalmente em uma trabalho anterior, podemos afirmar que para esse protocolo a confiabilidade também é de 100\%. 

\subsection{Nível de Abstração}

As operações de redes fornecidas pelo sistema são bastante abstratas, apesar de algumas exigirem o entendimento de ponteiros e espaços de memória, meros detalhes da linguagem C. O "diálogo" entre cliente e servidor é resolvido com mensagens, e as operações são tão abstratas quanto imaginamos: baseiam-se em processos de envio ({\it send()} e {\it recv()}). Mesmo operações exclusivas a servidores, como {\it bind()} e {\it listen()}, embora não sejam tão instintivas, uma vez entendido o protocolo, fazem sentido e tem boa abstração. 

Em suma, é possível perceber que toda a complexidade do protocolo está escondida atrás de operações simples.

\section{Conclusões}

No sistema cliente-servidor implementado neste projeto a informação que o servidor retém é a do seu banco de dados e todas as mensagens se baseiam em consultas {\it MySQL}. O tempo total está totalmente influenciado pelo tempo de acesso ao banco, como fica visível na operação de escrita.

Devido a característica desta aplicação apenas uma pequena parte do código do sistema realiza as operações de redes. Apesar disso, essas operações são carregadas pela sua abstração, onde toda a complexidade esta embutida apenas em um comando básico. A abstração condiz com o tempo de execução dessas operações, medidos no tempo de transmissão, que equivale a menos de um quarto do tempo total na maioria das operações.

A confiabilidade garantida pelo protocolo TCP utilizado na comunicação do sistema foi comprovada em todas as consultas. Um futuro trabalho será comparar esses resultados com um sistema que utiliza o protocolo UDP, onde a confiabilidade pode não ser tão perfeita.

\addcontentsline{toc}{section}{\protect\numberline{}Referências}

\singlespace
\bibliographystyle{unsrt}
\bibliography{relatorioRMI}

\end{document}
